%
% Please use latex to format this document.
%
\documentclass[11.5pt]{article}

\usepackage{amsmath}
\usepackage{amsfonts}
\usepackage{amssymb}
\usepackage{amsthm}
\usepackage{fancyhdr}
\usepackage{comment}
\usepackage{hyperref}

\usepackage[paper=letterpaper,
%includefoot, % Uncomment to put page number above margin
marginparwidth=1.2in,     % Length of section titles
marginparsep=.05in,       % Space between titles and text
margin=1in,
top=1.5in,
bmargin=1.0in,               % 1 inch margins
includemp]{geometry}


\marginparwidth 0pt
\oddsidemargin  0pt
\evensidemargin  0pt
\marginparsep 0pt

%\topmargin   -20pt

\textwidth   6.5 in
%\textheight  9 in

\pagestyle{fancy}


%\renewcommand{\baselinestretch}{2}
%%%%%%%%%%%%%%%%%%%%%%%%%%%%%%%%%%%%%%
% Double spacing                     %
%%%%%%%%%%%%%%%%%%%%%%%%%%%%%%%%%%%%%%

%\renewcommand{\baselinestretch}{2}

%%%%%%%%%%%%%%%%%%%%%%%%%%%%%%%%%%%%%%

%*************Declaration Section******************
\newcommand{\Z}{\bf{Z}}
\newcommand{\Q}{\bf{Q}}
\newcommand{\R}{\bf{R}}
\newcommand{\C}{\bf{C}}
\newcommand{\F}{\bf{F}}
\newcommand{\T}{\bf{T}}
\newcommand{\J}{\bf{J}}
\newcommand{\PP}{\bf{P}}
\newcommand{\RP}{{\noindent \bf Research problems:}}
\newcommand{\B}{{\noindent $\bullet$ \ }}
%\newcommand{\HH}{\rm{H}}
%\newcommand{\SS}{\rm{S}}
%\newcommand{\LL}{\rm{L}}
\newcommand{\Mid}{\mid\!\!}
\newcommand{\miD}{\!\!\mid}
\newcommand{\tensor}{\otimes}
\newcommand{\ra}{\rightarrow}
\font\cyr=wncyr10
\newcommand{\Sha}{\hbox{\cyr X}}
\newtheorem{lem}{Lemma}[section]
\newtheorem{cor}[lem]{Corollary}
\newtheorem{prop}[lem]{Proposition}
\newtheorem{conj}[lem]{Conjecture}
\newtheorem{thm}[lem]{Theorem}




\begin{document}
%%%%%%%%%%%%%%%%%%%%%%%%%%%%
% Topmatter                %
%%%%%%%%%%%%%%%%%%%%%%%%%%%%


\fancyhf{}
%\rhead{User ID: azadsalam@cse.buet.ac.bd}
\rhead{azadsalam2611@gmail.com}
\lhead{Abdus Salam Azad}
\lfoot{Personal Website: \url{https://sites.google.com/view/azadsalam2611}}
\rfoot{\thepage}

\begin{center}
{\LARGE \bf 
Diversity Essay}\\
%\vspace{0.1in}
%{\Large {Program: Ph.D. in Computer Science}}
% \\ azadsalam@cse.buet.ac.bd
\end{center}

%\\
%A
%\hfill
%Prof. Whatshisname

%\vspace{-0.15in}


I, Abdus Salam Azad, come from Bangladesh, the eighth happiest country in the world according to Happy Planet Index\footnote{http://happyplanetindex.org/countries/bangladesh}. With an area of just 147,570 square kilometres, it is a home to a stupendous 160 million people. Our country has a rich culture of literature, dance, drama, music, and the arts. It is also well known for its hospitality and festivity. Each year we celebrate Noboborso: the Bangla New Year, Pahela Falgun: a celebration to welcome spring, and many others. Being brought up in this country of diverse cultural background, a unique sense of culture has been instilled in my personality.

I completed my undergraduate and Master's from the Department of Computer Science (CSE), Bangladesh University of Engineering and Technology (BUET), the most prestigious engineering university in Bangladesh. Being a student of BUET I was privileged to have a campus life with hundreds of people from every corner of Bangladesh. This culture of diversity at my alma mater was taken care by its own ecosystem and I grew in this campus having the respect for people from different backgrounds and different beliefs. 

While an undergraduate student, I co-founded a voluntary organization, ``Engineering Students Association of Bangladesh'' (ESAB) and served as its Information and Communication Secretary (from September 2011 to November 2013). As a founding member, I still participate in active in its endeavors. Serving ESAB has given me the opportunity to experience a rare leadership journey. ESAB is the country's only Government registered student organization uniting all the bright minds to engage in country's development ventures. During its 6 year journey, ESAB has successfully transformed into a development organization driven by impact-oriented initiatives. It is focused on capacity development programs (e.g., seminar, workshop, and training programs), policy reforms activities (e.g., demanding additional budget allocation dedicated for research, financing youth innovations) as well as developing an international network. We are also forming an incubation center called `ESAB Innovation Center' with a vision to nurturing potential innovations for social goods. We've already partnered up with Prime Minister's Office, Ministry of Power, Energy and Mineral Resources, ICT Division and many national and international organizations. 

Due to my excellent academic records, I have also been serving as a lecturer at my own Alma Mater CSE, BUET since graduation. During this time I had the opportunity to connect with a wide variety of students through conducting lectures, grading assignments, career-counselling, brain-storming sessions, and so on. With time these experiences have increased my ability to analyze things from different perspectives and communicate effectively.  Bangladesh is currently undergoing a massive digitalization in many sectors, too and CSE, BUET is playing an important part in that process. As a faculty of CSE, BUET I have also contributed towards several of such projects. 

I think all the little experiences in different stages of my life has prepared me to have an open mind, patience, and respect for others, which have enabled me to mix well with different kinds of people. Being a representative of a country which has a diverse cultural background, I believe I will uniquely contribute towards the diversity of the Purdue University.

%Prof. Zhu's work on persistent homology including its application for a new representation of text is also intriguing. 

% %The idea of machine teaching seems very intriguing to me. I would also like to explore the use of Machine Teaching techniques for machine learning algorithms. For example, we can speed up the training by selecting batches intelligently. By machine teaching techniques, we find the representative subset of training examples for the current state of the model, and emphasize on other examples from the training set for the next batch to facilitate generalization---a concept similar to Boosting for Ensemble Learning. 

\end{document}


