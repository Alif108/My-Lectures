%
% Please use latex to format this document.
%
\documentclass[12pt]{article}

\usepackage{amsmath}
\usepackage{amsfonts}
\usepackage{amssymb}
\usepackage{amsthm}
\usepackage{fancyhdr}
\usepackage{comment}
\usepackage{hyperref}

\usepackage[paper=letterpaper,
%includefoot, % Uncomment to put page number above margin
marginparwidth=1.2in,     % Length of section titles
marginparsep=.05in,       % Space between titles and text
margin=1in,
top=1.5in,
bmargin=1in,               % 1 inch margins
includemp]{geometry}


\marginparwidth 0pt
\oddsidemargin  0pt
\evensidemargin  0pt
\marginparsep 0pt

%\topmargin   -20pt

\textwidth   6.5 in
%\textheight  9 in

\pagestyle{fancy}


%\renewcommand{\baselinestretch}{2}
%%%%%%%%%%%%%%%%%%%%%%%%%%%%%%%%%%%%%%
% Double spacing                     %
%%%%%%%%%%%%%%%%%%%%%%%%%%%%%%%%%%%%%%

%\renewcommand{\baselinestretch}{2}

%%%%%%%%%%%%%%%%%%%%%%%%%%%%%%%%%%%%%%

%*************Declaration Section******************
\newcommand{\Z}{\bf{Z}}
\newcommand{\Q}{\bf{Q}}
\newcommand{\R}{\bf{R}}
\newcommand{\C}{\bf{C}}
\newcommand{\F}{\bf{F}}
\newcommand{\T}{\bf{T}}
\newcommand{\J}{\bf{J}}
\newcommand{\PP}{\bf{P}}
\newcommand{\RP}{{\noindent \bf Research problems:}}
\newcommand{\B}{{\noindent $\bullet$ \ }}
%\newcommand{\HH}{\rm{H}}
%\newcommand{\SS}{\rm{S}}
%\newcommand{\LL}{\rm{L}}
\newcommand{\Mid}{\mid\!\!}
\newcommand{\miD}{\!\!\mid}
\newcommand{\tensor}{\otimes}
\newcommand{\ra}{\rightarrow}
\font\cyr=wncyr10
\newcommand{\Sha}{\hbox{\cyr X}}
\newtheorem{lem}{Lemma}[section]
\newtheorem{cor}[lem]{Corollary}
\newtheorem{prop}[lem]{Proposition}
\newtheorem{conj}[lem]{Conjecture}
\newtheorem{thm}[lem]{Theorem}




\begin{document}
%%%%%%%%%%%%%%%%%%%%%%%%%%%%
% Topmatter                %
%%%%%%%%%%%%%%%%%%%%%%%%%%%%


\fancyhf{}
%\rhead{User ID: azadsalam@cse.buet.ac.bd}
\rhead{azadsalam2611@gmail.com}
\lhead{Abdus Salam Azad}
\lfoot{Personal Website: \url{https://sites.google.com/view/azadsalam2611}}
\rfoot{\thepage}

\begin{center}
{\LARGE \bf 
Personal History Statement}\\
%\vspace{0.1in}
%{\Large {Program: Ph.D. in Computer Science}}
% \\ azadsalam@cse.buet.ac.bd
\end{center}

%\\
%A
%\hfill
%Prof. Whatshisname

%\vspace{0.1in}

When I reminisce about my childhood, the first thing that comes to my mind is the terrible colds I used to catch every now and then. At a very early stage of my life, I was fighting asthma. The best schools in my city were a bit far away from my home. My parents were concerned that the smoke and dust on the way to those schools could make my situation worse. Hence, I was admitted to a less competitive school in my neighbourhood.  In school, I used to outperform my classmates by a significant margin. However, my parents saved me from indulging in complacency. They always encouraged me to work harder, saying ``Your competition is only with you, none else.'' This has been the most profound lesson of my life that I continue to remind myself every day. With each sunrise, I still try to be a better version of myself.

Much later in life, I got admitted to BUET for undergraduate studies, the most prestigious engineering university in Bangladesh. Being the top-ranked university, BUET selects the most brilliant students of the country (around 1\% admission rate) through a very competitive admission test in each year. Among these admitted students, only the topmost students get the opportunity to study in the Department of Computer Science and Engineering (CSE). I have been fortunate to be the part of this department that regularly produces the country's top academics. Notably, I graduated \textit{summa cum laude} from the CSE department, which I believe has been possible through determination, perseverance, and a lot of hard work. After graduation, I joined CSE, BUET as a lecturer and since then I have been working there full-time. As part of my job, I have conducted several courses including Artificial Intelligence and Machine Learning. Currently, I also conduct a training course on practical Machine Learning for around 20 graduates from different universities of Bangladesh. This brief teaching experience has made me realize that to be able to serve academia in a more meaningful way, a deeper knowledge and practical experience are needed. And what can be a better way to prepare for this other than involving myself in the high-quality research of a Ph.D. program in a quest to create new knowledge.   

My interest in machine learning techniques started when I was a second year student. I was learning Java by writing small software and games. While implementing a color to grayscale conversion feature for an image editor, one thing puzzled my mind: How can I turn a grayscale picture back into a color one? After a few days of head scratching and internet surfing, I was puzzled to know about some machine learning techniques, which can solve these types of problems just by seeing examples. However, as I was yet to learn so many things before I could truly understand and implement such techniques, I continued to focus on other subjects. But my inclination towards machine learning and more generally towards artificial intelligence (AI) only increased with time.

I got my first opportunity to work with machine learning and related techniques in my junior year. I along with a group of researchers in CSE, BUET analyzed 40 years of historical weather data from different regions of Bangladesh to identify the trends in temperature and rainfall. We used a wide range of techniques, including clustering such as K-means, non-parametric trend tests such as Mann-Kendall and Sen's slope estimator, etc. We found a number of interesting insights, such as, over the years the maximum temperature of our country has significantly increased during June to November. In contrast, there have been no significant changes in rainfall. The results also indicate that in terms of temperature the eastern part of the country has faced more climatic changes than its western part. The findings of the study were published as a book chapter by Springer. 

Motivated by the outcome of this project, I focused on working with more fundamental machine learning and AI techniques. I pursued my undergraduate and Master's thesis on Evolutionary Algorithms---an exciting paradigm of algorithms which solve computation problems by mimicking evolution. I also explored other topics of machine learning and its application in various domains through my course-works, projects, and research collaborations. After completing my M.Sc., I have also studied deep learning techniques and become familiar with libraries such as Sci-kit Learn, WEKA, Tensor Flow, and Keras.

In addition to academic activities, while an undergraduate student, I co-founded \href{https://www.facebook.com/esab.bd}{Engineering Students Association of Bangladesh} (ESAB) and served as its Information and Communication Secretary for two years. While serving ESAB, I have experienced a rare leadership journey. This experience gave me an enormous scope to work with the best minds coming from universities all over the country. Also, Bangladesh is currently undergoing a massive digitalization in many sectors and CSE, BUET is playing an important part in that process. As a faculty of CSE, BUET I have also contributed towards several of such projects. These activities have also made me realize that the transferable skills gained by the research work at Ph.D. level can enable me to contribute more towards my nation.

On a personal note, my motherland Bangladesh is the eighth happiest country in the world according to Happy Planet Index 2016. It is a home to 160 million people, including more than 2 million indigenous people. Our country has a rich culture of literature, dance, drama, music, and arts. Being brought up in this country, a unique sense of culture has been instilled in my personality. Since my childhood, I have been making good friends from different religions and origins with an open mind and a respect for differing opinions. 

Finally, I think all the experience gained in different stages of my life has directed me towards this point when I believe my next step should be obtaining a higher degree to pursue my research, academic, and socio-economic goals. I strongly believe a Ph.D. will enable me to contribute more towards academia and help me to play a greater role in the development of my nation. I believe, I have grown a basic foundation to perform original research and I think the EECS department at UC Berkeley is one of those places where I can reach my full potential. Also, coming from a country with a rich culture of literature, music, and arts, I believe I will uniquely contribute towards the diversity of UC Berkeley.

\begin{comment}
In this age of information, most of the human knowledge can be found on the internet. However, when we have a query, we find relevant information using search engines and we need to go through the retrieved pages to find the answer. My goal is to build a system named ``Dr. Ask'', which will save the users from the trouble of going through multiple pages and read entire documents. For a given query, Dr. Ask will gather relevant information distributed in different formats (e.g., text, image, video, and maps) from a collection of sources (e.g., pages recommended by search engines) and prepare a concise answer. Challenges from different domains, such as natural language understanding and image understanding, arise while creating a system like Dr. Ask. With a fascination to work on the intersection between natural language and image understanding using machine learning techniques, I want to pursue my Ph.D. in the department of Computer Science at University of North Carolina at Chapel Hill. I believe a Ph.D. in my area of interest would be the foundational step for a research career in academia.

My interest in machine learning techniques started to grow during my second year of undergraduate studies at Bangladesh University of Engineering \& Technology (BUET). I was developing an image editor with different filters. While grayscale conversion of a color image is straightforward, I was puzzled with the reverse task. After a few days of research, I came to know that some machine learning techniques can solve this problem just by seeing examples. I was thrilled by my newfound knowledge and its endless possibilities. Since then I have been fascinated towards machine learning, and in general artificial intelligence. Afterwards, in my third year, I along with two of my classmates, developed an online portal providing an interactive interface for climate researchers to run different analysis on climate data. We used some standard statistical techniques, time series prediction models, and clustering techniques to study trends in temperature and rainfall data of Bangladesh. This data was collected by 37 weather stations for more than 40 years. The findings of the study were published as a book chapter by \href{http://link.springer.com/chapter/10.1007\%2F978-94-017-8642-3_3}{Springer Netherlands} in 2014. 

Later on, for my undergraduate thesis, I worked with \href{http://cse.buet.ac.bd/faculty/facdetail.php?id=mdmonirulislam}{Prof. Md. Monirul Islam}, one of the most prominent AI researchers in Bangladesh. I focused on solving Vehicle Routing Problems (VRP) and its different variants using Genetic Algorithms (GA). We studied the existing solutions for VRP and its variants, analyzed their strengths and weaknesses, and designed a basic GA framework for solving VRPs with multiple depots and periods. Our work got the first prize in the ``1st undergraduate thesis poster presentation, 2014, BUET.'' Even after graduation, I continued my work with Prof. Islam on VRPs. We have recently proposed a new variant of VRP with multiple depots and periods, which can provide cheaper solutions than the existing formulation. We have also designed a memetic algorithm to solve the variant. Our main focus has been designing a learning component that does not focus extensively on greediness, as too much greediness can lead to a premature convergence and often requires a population restart for further progress. We have proposed a stochastic memetic algorithm, focusing simultaneously on greediness and randomness, to maintain the balance between exploration and exploitation to avoid a premature convergence. Our work has been accepted for publication in the \href{http://ieeexplore.ieee.org/xpl/RecentIssue.jsp?punumber=6221036}{IEEE Transactions on Cybernetics}.

My current research works include a project with \href{http://cse.buet.ac.bd/faculty/facdetail.php?id=eunus}{Prof. Mohammed Eunus Ali}, where I have studied the problem of describing and retrieving images by natural language queries based on semantic similarities. I am developing a deep learning based method to solve this problem. %To understand the semantic of images, I am working on a two-phased method, where in the first phase the objects within the images will be identified and the semantic relations between the objects will be learned in the second phase. 
Currently, I am also working on the machine comprehension task under the supervision of Prof. Islam. This work is inherently related to ``Dr. Ask'' and focuses on the MCTest Dataset released by Microsoft Research. The task is to answer multiple choice questions based on fictitious short stories. We have modeled the problem as a game, where an agent upon given a question-answer pair traverses a graph which is constructed from the story by utilizing dependency parse trees and coreference resolution. We award points to such traversals based on their relevance with the given question-answer pair.  The agent is trained by reinforcement learning so that it can gain highest points given the correct question-answer pair. This is a work in progress which we plan to submit in EMNLP'17. 

% We calculate the relevance based on the probabilistic method proposed by Karthik Narasimhan and Prof. Regina Barzilay in their work on machine comprehension published at ACL 2015.

Since my graduation in 2014, I have been working as a lecturer in the Department of CSE, BUET. I have taught a number of theory and sessional courses including Machine Learning and Artificial Intelligence. I have also contributed towards different national projects, such as development of interactive digital textbooks and software testing for machine readable passports in 33 regional passport offices of Bangladesh. In addition to academic activities, I have co-founded \href{https://www.facebook.com/esab.bd}{Engineering Students Association of Bangladesh} (ESAB), first of its kind in Bangladesh, with a goal to bring all engineering students under one roof to work closely with the Government of Bangladesh for providing engineering solutions to local problems. 

I believe my knowledge, practical experience, and research interest make me a well-suited candidate for the prestigious department of Computer Science at University of North Carolina at Chapel Hill. I am deeply motivated by the works of Prof. Mohit Bansal. His state-of-the-art method on machine comprehension (ACL'15) has given me greater insight on the problem and new ideas to solve it.  His several other works, such as image and caption sequencing (EMNLP'16), visual question answering (EMNLP'16), and text-to-image coreference (CVPR'14), align with my research interest. The works of Prof. Tamara L. Berg and Alexander C. Berg in the intersection between natural language and vision deeply inspire me too. Their research papers on retrieving and generating natural language descriptions for images (IJCV'15) and on the Visual Madlibs dataset scintillate my interest. I strongly feel, the research environment of UNC-Chapel Hill is perfectly suitable for me to conduct research in my area of interest. Given an opportunity to work there, I believe I will be able to conduct novel research works with my determination, sincerity, and hard work to pursue a research-oriented career in academia.


\end{comment}


\end{document}


